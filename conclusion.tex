\section{Conclusion}
\label{sec:conclusion}






























































\iffalse
We proposed a new algorithm, \emph{EXPoSE}, to perform anomaly detection on very large-scale datasets and streams with concept drift. Although anomaly detection is a problem of central importance in many applications, only a few algorithms are scalable to the vast amount of data we are often confronted with.

The EXPoSE anomaly detection classifier calculates a score (the likelihood of a query point belonging to the class of normal data) using the inner product between a feature map and the kernel embedding of probability measures. The kernel embedding technique provides an efficient way to work with probability measures without the necessity to make assumptions about the underlying distributions.

Despite its simplicity EXPoSE obeys a \emph{linear} computational complexity for learning and can make predictions in \emph{constant} time while it requires only  \emph{constant} memory. 
When applied incrementally or online, a model update can also be performed in \emph{constant} time. We demonstrated that EXPoSE can be used as an efficient anomaly detection algorithm with the same predictive performance as the best state of the art methods while being significant faster than techniques with the same discriminant power.\cite{Habel_2007_IAG}
\fi