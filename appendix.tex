\appendix
%\counterwithin{figure}{section}
\section{Appendices}
\subsection{Meeting Minutes}

\subsubsection*{Minutes for January 30, 2017}

\textbf{Present:} 
{S.~Bancroft,
  A.~Bell,
  D.~Eden,
  J.~Farr,
  R.~Mercer,
  T.~Tzokov}
  
\large{Arduino Vs. NI DAQ}
\begin{itemize}
\item
Sampling rate of 1500 Hz for contact mic. Chris said Arduino Probably doesn't have high enough sampling rate to deal with this.

\item
Use National Instruments DAQ box - don't need to worry about live processing with Arduino for now.

\item
Previous group said sampling at 150 Hz is adequate, but Chris said that's for us to determine by using as high sampling frequencies as possible.

\item
Stop using Arduino - use National Instruments DAQ.

\item
Could possibly propose Arduino for a final design, if we determine that its sampling rate is high enough. But use NI DAQ box for the analysis as this has much higher sampling rate.

\item
Not providing hardware for a project, simply finding out which methods/algorithms work for data analysis - so doesn't matter that we're using National Instruments.
\end{itemize}

\large{Further Experimental Work}
\begin{itemize}
\item
Increase load on motor by placing rotor in water.

\item
Eliminate frequencies more than half the sampling frequency. Use low pass filter. 2nd or 4th order Butterworth filter to avoid aliasing.

\item
Should record data at higher sampling frequency than contact mic. Want to record at 3000 Hz or more to be more than twice the sampling frequency of the contact mic (Nyquist theorem).

\item
Overcharge a motor - see what happens when it gets close to failure.

\item
Data samples only need to be approximately 30 seconds long.

\item
How do you determine rotation speed of motor? - Place disk on end with hole and use a light gate to count rotations.

\end{itemize}

\large{Analysis}
\begin{itemize}
\item
Neural networks

\item
Average power spectra to get smooth curve and plot it. See where frequency peaks.

\item
Use chi-squared fitting to fit a peak to the data with error bars - find FWHM (full-width-half-maximum) and peak height etc.

\item
What are these measurements to good/bad motors. What is the `goodness' of fit for each parameter?

\item
Peak of frequency spectrum vs rotation speed of motor

\item
Goal is to return algorithms

\item
Try to quantify and understand data results before implementing neural networks.

\item
Contact Dr Hindmarch to get python data recording modules. Give continuous data acquisition using the national instruments DAQ box to give real time data.

\end{itemize}

\noindent\makebox[\linewidth]{\rule{0.7\textwidth}{0.4pt}} %\clearpage

\subsubsection*{Minutes for February 6, 2017}

\textbf{Present:} 
{S.~Bancroft,
  A.~Bell,
  D.~Eden,
  J.~Farr,
  R.~Mercer,
  T.~Tzokov}
  
\large{Recap}
\begin{itemize}
\item
Taken data with increasing voltages to look at stress on motors. Sampling rate of 3000 Hz.

\item
Going to use a 3D-printed fan to increase load on a motor by placing fan in water.

\item
Used a light gate to measure motor frequency using a ``Rotary Encoder''.
\end{itemize}

\large{To Do Next}
\begin{itemize}
\item
Should try measuring waveform of motor using lightgate and oscilloscope.

\item
Plot Lorentzian instead of Gaussian against peaks in fourier plots - better fit.

\item
Plot raw data against fitted function and also a function based on fitted parameters. Can assess if programming is correct and no mistakes are being made.

\item
Have y-displacement as a free parameter of the Lorentzian in case noise makes a base-level constant noise at all frequencies.

\item
Use chi-squared analysis on Lorentzian fitting.

\item
Use mechanical workshop to get metal cut with base and two sideplates to hold the motor.

\item
Find gear ratio of the motor.

\end{itemize}

\noindent\makebox[\linewidth]{\rule{0.7\textwidth}{0.4pt}} \clearpage

\subsubsection*{Minutes for February 13, 2017}

\textbf{Present:} 
{S.~Bancroft,
  A.~Bell,
  D.~Eden,
  J.~Farr,
  R.~Mercer,
  T.~Tzokov}
  
\large{Agenda}
\begin{itemize}
\item
Low pass/high pass filters, finding peaks and fitting the data.

\item
Using the fans.

\item
Recording data using DAQ box, plotting results and using statistical analysis to analyse them.
\end{itemize}

\large{Notes}
\begin{itemize}
\item
Make filter with software not hardware - more flexible and dynamic.

\item
Don't filter out noise when we don't know the cause - start taking data before motor runs to establish if noise appears with motor or if it's background. Quantify data before rejecting frequencies. Look at change in spectrum with and without motor running.

\item
Use ungeared motor with fan.

\item
See if spectrum degrades/peak widens with inducing failure.

\item
Take background into account when plotting Lorentzian.

\item
Bullet point sections of report.

\item
Potentially look at using a microphone as cheaper alternative if there is time.

\item
Fourier plot use semilog y-axis.
\end{itemize}


\large{To Do Next}
\begin{itemize}
\item
Order the lightgate.

\item
Use thicker fans and take data from the changing load.

\item
Induce failures in different ways and get lots of data.
\end{itemize}

\noindent\makebox[\linewidth]{\rule{0.7\textwidth}{0.4pt}} \clearpage

\subsubsection*{Minutes for February 20, 2017}

\textbf{Present:} 
{S.~Bancroft,
  A.~Bell,
  J.~Farr,
  R.~Mercer,
  T.~Tzokov}
  
\large{Notes}
\begin{itemize}
\item
Look in both domains for statistical tests.

\item
Probability of false positives - Gaussian, probability per time scale (e.g. day).

\item
Table for anomaly detection methods.

\item
Citrus degreaser from bike shop for motors.

%\item
%Optimisation of code to give factor of 10 increase to speed.

\end{itemize}


\large{To Do Next}
\begin{itemize}
\item
Let Chris know when Owen will come this week.

\item
Look at adding errors to Lorentzian.

\item
Look into damaging motor brushes.

\item
Order lead acid battery as alternative power supply, remember fuse and charger.

\item
Begin splitting up report and adding placeholder figures.

\item
Prepare PowerPoint with some graphs for meeting with Owen.
\end{itemize}

\noindent\makebox[\linewidth]{\rule{0.7\textwidth}{0.4pt}} %\clearpage
\vspace{-5ex}
\subsubsection*{Minutes for February 27, 2017}

\textbf{Present:} 
{S.~Bancroft,
  A.~Bell,
  D.~Eden,
  J.~Farr,
  R.~Mercer,
  T.~Tzokov}
  
\large{Notes}
\begin{itemize}
\item
Report deadline 17/03/17 (should be finished week prior to allow presentation practice).

\item
Owen to travel to Durham for 30 minute presentation 17/03/17 or when convenient for him.

\end{itemize}

\large{To Do Next}
\begin{itemize}
\item
Leave the motor running once brushes have been sanded to observe if the motor repairs itself.

\item
Finish data collection in roughly 1.5 weeks time.

\item
Look for coincidence in anomaly metrics, scatter plot the anomaly metrics against each other and identify clusters.

\item
Present a description of a plot and an example to Chris later this week to verify we know what metrics are doing.

\item
Look into wind turbine failure modes, possible avenue of comparison.

\item
Organise a date for presentation with Prof. Chadwick, Chris and Owen.

\end{itemize}

\noindent\makebox[\linewidth]{\rule{0.7\textwidth}{0.4pt}} \clearpage
