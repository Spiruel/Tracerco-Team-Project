	\title{\normalfont\spacedallcaps{Anomaly Detection in the time series data of motor vibrations}}
	\author{\spacedlowsmallcaps{Samuel Bancroft, Alexander Bell, Danielle Eden,}\\
		\spacedlowsmallcaps{Jake Farr, Robert Mercer \& Teodor Tzokov}
		\\ {\textit{Department of Physics, Durham University}}}
	\date{\today}
	\titlepic{\includegraphics[width=0.3\textwidth]{formatting/DU_2-col_sml.eps}}
\maketitle
	
\begin{center}
	\begin{tcolorbox}[colback=white,width=\textwidth,colframe=white]
		\section*{\large Abstract}
		\small
		%Outline, Methods, Outcomes, Conclusions. 
		% Analysis kids can make this more specific in the outcomes/conclusions
		Throughout this investigation, we have worked on using analysis of the vibrations of a motor to detect an impending failure. This has been done by inducing a range of failures through different failure modes in the motor, such as overheating or overvoltaging, whilst monitoring its vibrations through a contact microphone. A range of anomaly detection methods could then be applied to this data to find errors, working under the assumption that anomalous data will be indicative of a motor failure. A number of different anomaly detection methods have been trialled, some of which have been discarded for reasons that will be discussed. It has been determined that implementing a range of anomaly detection methods simultaneously is the best way to detect the majority of failures whilst avoiding false positives.
	\end{tcolorbox}
\end{center}
		
\newpage

\tableofcontents
\clearpage
\listoffigures
\listoftables
\clearpage
