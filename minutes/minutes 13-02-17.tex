\documentclass[11pt]{meetingmins}

\setcommittee{TracerCo Team Project Term 2 - Meeting Minutes}

\setmembers{
  S.~Bancroft,
  A.~Bell,
  D.~Eden,
  J.~Farr,
  R.~Mercer,
  T.~Tzokov
}

\setdate{February 13, 2017}

\setpresent{
  S.~Bancroft,
  A.~Bell,
  D.~Eden,
  J.~Farr,
  R.~Mercer,
  T.~Tzokov
}

\begin{document}
\maketitle


\section{Agenda}
\begin{items}
\item
Low pass/high pass filters, finding peaks and fitting the data.

\item
Using the fans.

\item
Recording data using DAQ box, plotting results and using statistical analysis to analyse them.
\end{items}

\section{Notes}
\begin{items}
\item
Make filter with software not hardware - more flexible and dynamic.

\item
Don't filter out noise when we don't know the cause - start taking data before motor runs to establish if noise appears with motor or if it's background. Quantify data before rejecting frequencies. Look at change in spectrum with and without motor running.

\item
Use ungeared motor with fan.

\item
See if spectrum degrades/peak widens with inducing failure.

\item
Take background into account when plotting Lorentzian.

\item
Bullet point sections of report.

\item
Potentially look at using a microphone as cheaper alternative if there is time.

\item
Fourier plot use semilog y-axis.
\end{items}


\section{To Do Next}
\begin{items}
\item
Order the lightgate.

\item
Use thicker fans and take data from the changing load.

\item
Induce failures in different ways and get lots of data.
\end{items}

\end{document}