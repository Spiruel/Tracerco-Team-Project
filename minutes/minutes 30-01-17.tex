\documentclass[11pt]{meetingmins}

\setcommittee{TracerCo Team Project Term 2 - Meeting Minutes}

\setmembers{
  S.~Bancroft,
  A.~Bell,
  D.~Eden,
  J.~Farr,
  R.~Mercer,
  T.~Tzokov
}

\setdate{January 30, 2017}

\setpresent{
  S.~Bancroft,
  A.~Bell,
  D.~Eden,
  J.~Farr,
  R.~Mercer,
  T.~Tzokov
}


\begin{document}
\maketitle


\section{Arduino Vs. NI DAQ}
\begin{items}
\item
Sampling rate of 1500Hz for contact mic. Chris said Arduino Probably doesn't have high enough sampling rate to deal with this.

\item
Use National Instruments DAQ box - don't need to worry about live processing with Arduino for now.

\item
Previous group said sampling at 150Hz is adequate, but Chris said that's for us to determine by using as high sampling frequencies as possible.

\item
Stop using Arduino - use National Instruments DAQ.

\item
Could possibly propose Arduino for a final design, if we determine that its sampling rate is high enough. But use NI DAQ box for the analysis as this has much higher sampling rate.

\item
Not providing hardware for a project, simply finding out which methods/algorithms work for data analysis - so doesn't matter that we're using National Instruments.
\end{items}

\section{Further Experimental Work}
\begin{items}
\item
Increase load on motor by placing rotor in water.

\item
Eliminate frequencies more than half the sampling frequency. Use low pass filter. $2nd$ or $4th$ order Butterworth filter to avoid aliasing.

\item
Should record data at higher sampling frequency than contact mic. Want to record at 3000Hz or more to be more than twice the sampling frequency of the contact mic (Nyqist theorem).

\item
Overcharge a motor - see what happens when it gets close to failure.

\item
Data samples only need to be approximately 30 seconds long.

\item
How do you determine rotation speed of motor? - Place disk on end with hole and use a light gate to count rotations.

\end{items}

\section{Analysis}
\begin{items}
\item
Neural networks

\item
Average power spectra to get smooth curve and plot it. See where frequency peaks.

\item
Use chi-squared fitting to fit a peak to the data with error bars - find FWHM (full-width-half-maximum) and peak hight etc.

\item
What are these measurements to good/bad motors. What is the 'goodness' of fit for each parameter?

\item
Peak of frequency spectrum vs rotation speed of motor

\item
Goal is to return algorithms

\item
Try to quantify and understand data results before implementing neural networks.

\item
Contact Dr Hindmarch to get python data recording modules. Give continuous data acquisition using the national instruments DAQ box to give real time data.

\end{items}


\end{document}