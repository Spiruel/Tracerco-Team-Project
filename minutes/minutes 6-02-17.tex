\documentclass[11pt]{meetingmins}

\setcommittee{TracerCo Team Project Term 2 - Meeting Minutes}

\setmembers{
  S.~Bancroft,
  A.~Bell,
  D.~Eden,
  J.~Farr,
  R.~Mercer,
  T.~Tzokov
}

\setdate{February 6, 2017}

\setpresent{
  S.~Bancroft,
  A.~Bell,
  D.~Eden,
  J.~Farr,
  R.~Mercer,
  T.~Tzokov
}

\begin{document}
\maketitle


\section{Recap}
\begin{items}
\item
Taken data with increasing voltages to look at stress on motors. Sampling rate of 3000Hz.

\item
Going to use a 3D-printed fan to increase load on a motor by placing fan in water.

\item
Used a light gate to measure motor frequency using a "Rotary Encoder".
\end{items}

\section{To Do Next}
\begin{items}
\item
Should try measuring waveform of motor using lightgate and oscilloscope.

\item
Plot Lorentzian instead of Gaussian against peaks in fourier plots - better fit.

\item
Plot raw data against fitted function and also a function based on fitted parameters. Can assess if programming is correct and no mistakes are being made.

\item
Have y-displacement as a free parameter of the Lorentzian in case noise makes a base-level constant noise at all frequencies.

\item
Use chi-squared analysis on Lorentzian fitting.

\item
Use mechanical workshop to get metal cut with base and two sideplates to hold the motor.

\item
Find gear ratio of the motor.


\end{items}
\end{document}